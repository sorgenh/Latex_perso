\documentclass[10pt,a4paper]{moderncv}
\moderncvstyle{classic}                             % style options are 'casual' (default), 'classic', 'oldstyle' and 'banking'
\moderncvcolor{blue} 
\usepackage[utf8]{inputenc}
\usepackage[scale=0.86]{geometry}

\firstname{Hadrien}
\familyname{Négros}
%\title{\LARGE Etudiant en Informatique}             
\address{13 rue Duc}{75018 Paris}{}  
\mobile{07 86 51 20 36}    
\email{hadrien.negros@gmail.com}   
%\homepage{www.hadriennegros.fr}                
\extrainfo{30 ans}                   
\begin{document}
\maketitle

\section{Expériences}
\cventry{Jan 2019 -- Juin 2019}{Senior Data Engineer}{RelevanC}{}{}{
	Administration d'un cluster MapR (20 noeuds). Création de workflows pyspark orchestrés par airflow.
	Gestion de l'infrastructure Data (\textasciitilde40 serveurs dédiés). Création de playbooks Ansible pour l’installation et la maintenance des serveurs.
	Installation de 2 clusters ElasticSearch.
	Installation d'un cluster Kubernetes on premise.
	Création d'une CI/CD via gitlab pour les projets Data, avec déploiement sur Kubernetes.
}
\cventry{Dec 2014 -- Dec 2018}{Data Engineer}{EDF}{}{}{
\textbf{Pilote de projet} Prise en charge de la migration de l'outil de tracking web d'EDF (Comscore $ \longrightarrow  $ Google Analytics) pour nos besoins analytiques internes: refonte complète du workflow d'intégration et évolution des DataMarts. Interface avec les APIs GCP \textit{(Python)}. Historisation, réconcilliation avec les données Comscore et rapprochement avec les données client \textit{(Hive)}. Transformations et diffusion de l'identifiant client \textit{(PySpark)}.\\
\textbf{Expert technique Hadoop pour les POCs de la Direction Numérique :} Mise en place d'un job Spark en java pour la génération d'un très grand nombre de graphiques personnalisés destinés à être envoyés par mail aux clients particulier. (passé en production) Création d'un job de classification de données textuelles avec la MLLib de Spark. (abandonné)\\
\textbf{Développeur Hadoop pour le \textit{DataLake} de la direction commerce d'EDF :} Divers travaux d'intégration de données necessitant des developpements Java (Map Reduce et Spark), Shell et Python pour la mise à disposition et l'enrichissement de données sur Hive et HBase. Ordonancement de ces travaux avec Oozie.
Mise en place d'une stack ELK pour le monitoring applicatif.}
\cventry{depuis Sept 2018}{Enseignant Vacataire}{Telecom ParisTech}{}{}{Enseignement du module "Introduction au Framework Hadoop" du mastère Big Data: Gestion et Analyse des Données Massives.}
\cventry{depuis Fév 2018}{Enseignant Vacataire}{Université Paris 1 Panthéon-Sorbonne}{}{}{Construction et enseignement du cours "Introduction aux Big Data et Data Mining" du Master 2 "Management des systèmes d'information".}
\cventry{Mar-Sep 2014}{IT Specialist Intern}{IBM}{}{}{Exploitation des outils Big Data d'IBM pour faciliter la résolution des incidents dans un Cloud Data Center.
Création d'une application BigInsights (La distribution Hadoop d'IBM) pour l'analyse de tickets d'incidents en langage naturel.
Mise en place d'outils d'analyse de logs: Extraction, indexation, root cause analysis...}
\cventry{Sep-Déc 2013}{Enseignant Vacataire}{Université Montpellier 2}{}{}{Encadrement des Travaux Pratiques de l'Unité d'Enseignement "Concepts de base en informatique".}
\section{Diplômes et Études}
\cventry{2014}{Master Données, Connaissances et Langage naturel}{Université Montpellier 2}{}{}{Spécialisation dans les domaines de l'ingénierie des connaissances, du traitement automatique du langage naturel et de la fouille de données. }
\cventry{2011}{Licence d'Informatique}{Université Montpellier 2}{}{}{}
%\cventry{2010}{Année de licence d'Informatique en Erasmus}{Trinity College Dublin}{}{}{}
%\cventry{2006}{Baccalauréat Scientifique}{Lycée du Mont-Blanc}{}{Mention Bien}{}

%\cventry{été 2013}{Conception d'une solution d'accès au divertissement}{Ginov Solutions}{Paris 1er}{}{Conception et implémentation d'un système d'extraction de mots clés basé sur des patrons d'étiquettes morpho-syntaxiques.
%\\ Modélisation de la base de données sur laquelle va se baser l'application.
%\\ Création d'un petit outil en C\# permettant de peupler cette base à partir de fichiers XML dont la structure est connue.
%\\Modélisation et implémentation d'un moteur de recherche simple prenant en compte la connaissance utilisateur.{}}
%\cventry{été 2012}{Intérim informatique}{Quechua (groupe Oxylane)}{Passy}{}{Divers travaux de création de documents et de formation basique sur Microsoft Excel{}}


%\section{Projets Universitaires}
%\cventry{Janvier-Mai 2013}{Travail d'étude et de recherche}{Comparaison d'étiqueteurs morpho-syntaxiques}{}{encadré par Jacques Chauché, Violaine Prince et Mathieu Roche}{Développement d'une nouvelle méthode de comparaison entre des étiqueteurs morpho-syntaxiques.{}}
%\cventry{Mars-Mai 2013}{Extraction de connaissances à partir de données}{Projet classification de données textuelles}{}{}{Étude de plusieurs algorithmes de classification sur un corpus de résumés de films.{}}

%\clearpage

\section{Compétences}
\cvcomputer{Langages}{Python, Java, R, Bash...}{Big Data}{Hadoop \emph{(Mapr, Hortonworks)}, HDFS, YARN, Spark, Hive, Oozie, Hbase, Kafka, Nifi, Airflow}
\cvcomputer{DevOps}{Git, Docker, Kubernetes, CI/CD Gitlab}{Data Science}{Connaissances en Text-Mining et Machine Learning}
\cvcomputer{Base de données}{Modèles relationnels (Oracle, SQLServer, PostGres), NoSQL (ElasticSearch, Neo4j, HBase)}{Autre}{Administration Linux. Architecture des systèmes d'information.}

%\subsection{Langues}


\section{Autre}
\cvcomputer{Anglais}{Courant}{Permis B}{Obtenu en 2007}
%\cvitem{Permis B}{Obtenu en 2007}
%\subsection{Centres d'intérêt}
%\cvline{Piano}{Élève pianiste pendant 8 ans, pratique régulière}{}{}
%\cvline{Sport}{Pratique du ski, du snowboard, de la natation}{}{}
%\cvline{Joueur de go}{(jeu de stratégie chinois)}{}{}
\end{document}
