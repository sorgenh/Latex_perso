\documentclass[10pt,a4paper]{moderncv}
\moderncvstyle{classic}                             % style options are 'casual' (default), 'classic', 'oldstyle' and 'banking'
\moderncvcolor{blue} 
\usepackage[utf8]{inputenc}
\usepackage[scale=0.825]{geometry}

\firstname{Hadrien}
\familyname{Négros}
%\title{\LARGE Etudiant en Informatique}             
\address{13 rue Duc}{75018 Paris}{}  
\mobile{07 86 51 20 36}    
\email{sorgenh@gmail.com}   
%\homepage{www.hadriennegros.fr}                
\extrainfo{27 ans}                   
\begin{document}
\maketitle
\section{Diplômes et Études}
\cventry{2014}{Master Données, Connaissances et Langage naturel}{Université Montpellier 2}{}{}{Spécialisation dans les domaines de l'ingénierie des connaissances, du traitement automatique du langage naturel et de la fouille de données. }
\cventry{2011}{Licence d'Informatique}{Université Montpellier 2}{}{}{}
%\cventry{2010}{Année de licence d'Informatique en Erasmus}{Trinity College Dublin}{}{}{}
%\cventry{2006}{Baccalauréat Scientifique}{Lycée du Mont-Blanc}{}{Mention Bien}{}

\section{Expériences}
\cventry{depuis Dec 2014}{Analyste Réalisateur - Big Data}{EDF}{}{}{Développement sur Hadoop pour le \textit{DataLake} de la direction commerce d'EDF. Divers travaux d'intégration de données necessitant des developpements Java (Map Reduce et Spark), Shell et Scala pour la mise à disposition de données sur Hive et HBase. Ordonancement de ces travaux avec Oozie.
Mise en place d'une stack ELK pour le monitoring applicatif.
Gestion de projet sur des POCs autour de la BI et de Hadoop.}
\cventry{Mar-Sep 2014}{IT Specialist Intern}{IBM}{}{}{Exploitation des outils Big Data d'IBM pour faciliter la résolution des incidents dans un Cloud Data Center.
Création d'une application BigInsights (La distribution Hadoop d'IBM) pour l'analyse de tickets d'incidents en langage naturel.
Mise en place d'outils d'analyse de logs: Extraction, indexation, root cause analysis...}
\cventry{Sep-Déc 2013}{Vacataire d'enseignement}{Université Montpellier 2}{}{}{Encadrement des Travaux Pratiques de l'Unité d'Enseignement "Concepts de base en informatique".}
\cventry{été 2013}{Conception d'une solution d'accès au divertissement}{Ginov Solutions}{Paris 1er}{}{Conception et implémentation d'un système d'extraction de mots clés basé sur des patrons d'étiquettes morpho-syntaxiques.\\ Modélisation de la base de données sur laquelle va se baser l'application.\\ Création d'un petit outil en C\# permettant de peupler cette base à partir de fichiers XML dont la structure est connue.
\\Modélisation et implémentation d'un moteur de recherche simple prenant en compte la connaissance utilisateur.{}}
%\cventry{été 2012}{Intérim informatique}{Quechua (groupe Oxylane)}{Passy}{}{Divers travaux de création de documents et de formation basique sur Microsoft Excel{}}


\section{Projets Universitaires}
\cventry{Janvier-Mai 2013}{Travail d'étude et de recherche}{Comparaison d'étiqueteurs morpho-syntaxiques}{}{encadré par Jacques Chauché, Violaine Prince et Mathieu Roche}{Développement d'une nouvelle méthode de comparaison entre des étiqueteurs morpho-syntaxiques.{}}
\cventry{Mars-Mai 2013}{Extraction de connaissances à partir de données}{Projet classification de données textuelles}{}{}{Étude de plusieurs algorithmes de classification sur un corpus de résumés de films.{}}

%\clearpage

\section{Compétences}
\cvcomputer{Langages}{Java, C\#, Python, Scala, Bash...}{IDE}{Vim, Eclipse, IntelliJ}
\cvcomputer{Conception\\Modèlisation}{UML, automates, graphes, réseaux de Pétri}{Big Data}{Hadoop (HDFS, YARN), Spark, Hive, Oozie, Hbase}
\cvcomputer{Base de données}{Modèles relationnels (Oracle, SQLServer), NoSQL (ElasticSearch, HBase, Hive, Neo4j)}{Autre}{\LaTeX. Architecture des systèmes d'information.}

%\subsection{Langues}


\section{Autre}
\cvcomputer{Anglais}{Courant}{Permis B}{Obtenu en 2007}
%\cvitem{Permis B}{Obtenu en 2007}
%\subsection{Centres d'intérêt}
%\cvline{Piano}{Élève pianiste pendant 8 ans, pratique régulière}{}{}
%\cvline{Sport}{Pratique du ski, du snowboard, de la natation}{}{}
%\cvline{Joueur de go}{(jeu de stratégie chinois)}{}{}
\end{document}
